\job
{Summer 2013} {}
{Amazon.com} {https://amazon.com}
{Software Development Engineer Intern}
{Designed and implemented a service for receiving and storing activity metrics for internal tools.}

\job
{2014} {2016}
{AWS Silk} {http://docs.aws.amazon.com/silk/latest/developerguide/introduction.html}
{Software Development Engineer I-II}
{Designed and built customer facing features, viewed millions of times a day, for the Silk browser.}

\job
{2016} {2017}
{Amazon Shopping App} {https://amazon.com/gp/feature.html?docId=1000625601}
{Software Development Engineer II}
{During my time on the shopping app team I dug into the Gradle build system and the ecosystem of custom plugins and build scripts within Amazon spanning hundreds of code repositories. After a deep investigation of much code, both internal and external, I worked my way through upgrading much of the internal tooling to the latest versions of the public build tools. While this work was frustrating at times, it was extremely rewarding to know how many other developers would benefit from my learnings through reduced build complexity and duration as well as increased understanding of how the build works through new documentation and internally recorded talks. Since leaving this team, I have maintained contact with engineers and have aided the internal build tooling team ramp up an engineer dedicated full time to the types of tasks I was working on.}

\job
{2017} {Present}
{AWS} {https://aws.amazon.com}
{Software Development Engineer II}
{After my stint in the Shopping App org, I was tapped to join a new AWS service (currently unreleased) right at the beginning. Since then I have been leading feature investigations and architecture designs, building critical components, and scaling many different facets of the service with my colleagues. After going in with little domain expertise, I have learned many new technical skills and developed a more comprehensive background.}